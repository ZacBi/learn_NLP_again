\documentclass{article}
\usepackage{listings}
\usepackage{xeCJK}
\usepackage{amsmath}
\usepackage{amssymb}
\usepackage{datetime}
\usepackage{indentfirst}
\usepackage{color}
\setlength{\parindent}{2em}
\usepackage[a4paper, left = 2cm, right = 2cm]{geometry}
\usepackage[colorlinks, urlcolor=blue]{hyperref}

% renew
\renewcommand{\today}{\number\year 年 \number\month 月 \number\day 日}


\begin{document}

\begin{titlepage}
    \title{Optimization in Deeplearning}
    \author{\href{https://github.com/ZacBi}{Zac Bi}}
    \maketitle
    \pagestyle{empty}
\end{titlepage}

\section{Why we need LSTM and GRU?}

\subsection{Gradient vanish and explosion}

梯度爆炸就不说了, 训练过程中表现为tensor的某个值为NaN, 造成无法计算, 一般的解决方法是gradient clip(梯度裁剪)
主要讨论的是gradient vanish, 说实话, 这个问题已经被讲烂了, 但是看了好多文章, 都是从矩阵的某个值讨论, 或者
直接从intuition的角度去讨论, 实在让人费解: 明明有严格的数学推导, 为啥不讲呢, 是Jaccobian Matrix飘了, 还是Spectral norm不够骚了?

下面这段推导和分析大部分来自这篇\href{https://arxiv.org/pdf/1211.5063.pdf}{paper}.
数学推导都需要定义一些annotation:
\begin{align*}
     & \textbf{E} \in \textbf{R}: \text{整体损失, 可理解为每一时间步的损失加和;} \\
     & \textbf{W} \in \textbf{R}^{d \times d}: \text{隐层单元的权重;}            \\
     & \textbf{h} \in \textbf{R}^d: \text{隐层状态}                              \\
     & \textbf{y} \in \textbf{R}^d: \text{输出;}                                 \\
     & \textbf{u} \in \textbf{R}^d: \text{仿射层;}                               \\
     & diag: \text{对角矩阵.}
\end{align*}

我们知道BPTT(back propagation through time)在vallina RNN中可以写成:
\begin{align}
    \frac{\partial \textbf{E}}{\partial \textbf{W}} = \sum_{t=1}^{T} \frac{\partial \textbf{E}_t}{\partial \textbf{W}}
\end{align}

其中$\frac{\partial \textbf{E}_t}{\partial \textbf{W}}$又可以写成:
\begin{align}
    \frac{\partial \textbf{E}_t}{\partial \textbf{W}} & = \sum_{k=1}^{t} \frac{\partial \textbf{E}_t}{\textbf{y}_t}
    \frac{\partial \textbf{y}_t}{\partial \textbf{h}_t} \frac{\partial \textbf{h}_t}{\partial \textbf{h}_k}
    \frac{\partial \textbf{h}_k}{\partial \textbf{W}}
\end{align}

其中$\frac{\partial \textbf{h}_t}{\partial \textbf{h}_k}$又又可以写成
\begin{align}
    \frac{\partial \textbf{h}_t}{\partial \textbf{h}_k} & = \prod_{j=k+1}^{t} \frac{\partial \textbf{h}_j}{\partial \textbf{h}_{j-1}}  \\
                                                        & = \prod_{j=k+1}^{t} \frac{\partial \textbf{h}_j}{\partial f(\textbf{u}_{j})}
    \frac{\partial f(\textbf{u}_{j})}{\partial \textbf{h}_{j-1}}                                                                       \\
                                                        & = \prod_{j=k+1}^{t} diag(f'(\textbf{u}_{j})) \cdot \textbf{W}
\end{align}

可能对$diag(f'(\textbf{u}_{j}))$有点疑问, 为啥这样写呢? 因为Jacobian matrix是这么写的:
\begin{equation*}
    \frac{\partial \textbf{h}_t}{\partial \textbf{u}_t} =
    \begin{bmatrix}
        \frac{\partial \textbf{h}_{t,1}}{\partial \textbf{u}_{t,1}} & \cdots        & \frac{\partial \textbf{h}_{t,1}}{\partial \textbf{u}_{t,d}} \\
        \vdots                                                      & \ddots \vdots                                                               \\
        \frac{\partial \textbf{h}_{t,1}}{\partial \textbf{u}_{t,d}} & \cdots        & \frac{\partial \textbf{h}_{t,d}}{\partial \textbf{u}_{t,d}}
    \end{bmatrix}
\end{equation*}

注意Jacobian矩阵的某一个元素$\left[\frac{\partial {h}_{t}}{\partial {u}_{t}}\right]_{i,j} = \left[\frac{\partial {h}_{t,i}}{\partial {u}_{t,j}}\right]$
只有当$i == j$ 时该元素才有值, 且值为$f'(z_{t,j})$.
暂时需要记住: (1)矩阵的乘积可以通过矩阵的谱范数进行近似, 谱范数也就是矩阵的L-2范数, 其值为矩阵最大的奇异值.
(2)由谱范数的相容性性质, 有||AB|| $\leq$ ||A|| ||B||
\begin{align*}
    \begin{Vmatrix}
        \frac{\partial \textbf{h}_t}{\partial \textbf{h}_{t-1}}
    \end{Vmatrix}
    \leq
    \begin{Vmatrix}
        \textbf{W}
    \end{Vmatrix}
    \begin{Vmatrix}
        diag(f'( \textbf{u}_t ))
    \end{Vmatrix}
    \leq \beta_h \beta_w
\end{align*}

代入到(3)式中有:
\begin{align*}
    \begin{Vmatrix}
        \frac{\partial \textbf{h}_t}{\partial \textbf{h}_{k}}
    \end{Vmatrix}
    =
    \begin{Vmatrix}
        \prod_{j=k+1}{t} \frac{\partial \textbf{h}_j}{\partial \textbf{h}_{j-1}}
    \end{Vmatrix}
    \leq (\beta_h \beta_w)^{t-k}
\end{align*}

一般来说, $\beta_h$是小于等于1的(sigmoid, tanh等), 主要考虑的是$\beta_W$. 当$(\beta_h \beta_w)$小于1时,
$(\beta_h \beta_w)^{t-k}$ 因为其指数级的形式将远小于1. 而对于当前时间步越远的输入, 其对当前时间步的梯度贡献越小, 就无法建立有效的长距离依赖.
实践之中梯度消失更常见, 相对于梯度爆炸来说也没有特别有效的解决方法(指完全解决, 毕竟梯度爆炸可以用梯度裁剪保留一部分梯度), 所以一般来说更侧重研究梯度消失.

\subsection{LSTM解决了Gradient vanish吗?}
答案是没有, 但是一定程度上缓解了梯度消失. 需要对梯度流进行计算.
概括一下是因为LSTM中的加法和门机制使得梯度容易控制. 因为gate是在每个时间步去学习的, 所以在每个时间步的递归梯度
可以通过那一个时间步学习到的门来控制梯度的消失或者保留. 具体的推导先留个坑吧. 可以先看这篇.

\section{归一化(Normalization)}

看过了深度学习500问, 总感觉写的很不明确, 还是想把归一化给理一下。
补充: 最近把知乎那篇Norm总结又翻了一下, 发现讲的东西其实都是论文的翻译, 以前见识少,
吃了文化的亏(逃

\subsection{为什么需要归一化?}


我们知道, 归一化是将我们的特征/向量中的分量(scalar)统一到一个区间, 比如[0, 1]或者[-1, 1]。
但是为什么要做归一化呢?很常见地解释之一是归一化使得使用梯度下降法求解问题的模型, 在损失函数等高线上的轮廓可能表现为均向性,
但是由于特征分量的量纲不同, 比如输入特征$\textbf{x} \in R^d$, $\textbf{x}_1$取值范围为[0, 3],
$\textbf{x}_2$的取值范围为[0, 1000], 就有可能出现即使梯度等高线是均向的, 但梯度仍以'之'字形在移动, 收敛速度较慢,
但对各个分量统一量纲后, 可以使得梯度方向尽可能指向最优点. 注意上述是建立在\textbf{等高线均向但特征向量各分量量纲不一}的情况下讨论的.

对于上面这种说法, 我是加了一点补充的, 因为还没有看到相关的paper或者book专门说明权重或者数据输入的量纲不同对于梯度等高线的影响.
仍然需要进一步验证. 需要注意的是, 对于数据和权重的归一化能极大优化训练的效果, 已经被证明可行的, 后面BN, LN, WN, GN, IN等会提到这些.

\subsection{归一化的一般形式}

第一种是线性函数归一化(Min-max scaling):
\begin{align}
    X_{\text{norm}} = \frac{X - X_{\text{min}}}{X_{\text{max}} - X_{\text{min}}}
\end{align}

第二种是零均值归一化(Z-score norm):
\begin{align}
    X_{\text{norm}} = \frac{ X - \mathbb{E}[X]}{ \sqrt{\mathbb{D}[X]}}
\end{align}

上面两个公式暂时忘记, 用paper里面总结的一般公式. 粗体小写表示vector, 粗体大写表示matrix,
小写表scalar, $\odot$ 表Hadamard product.
其中\textbf{g}表示放缩因子, $\sigma$表示std, $\mu$表示期望, \textbf{W}本层权重, \textbf{x}本层输入.
\begin{align}
    \textbf{h} = \textbf{f} (\frac{\textbf{g}}{\sigma} \odot (\textbf{Wx} - \mu) + \textbf{b})
\end{align}

\subsection{\href{http://arxiv-web1.library.cornell.edu/abs/1502.03167v1}{批归一化(Batch Normaliz)}}

BN应该是面试的时候最喜欢问的. 首先讲一下BN.

\subsubsection{BN解决了什么问题?}

paper中提到的是Internal covariant shift问题, 需要先来解释一下covariate shift是啥.
当训练数据集和测试数据集的分布不一致的时候, 训练出来的模型就不一定在测试集上表现的很好, 这种现象就是covariate shift.
用概率来表示, 就是两个数据集的条件分布一致但边缘分布不一致, 即$P_s(Y|X) == P_t(Y|X)$但$P_s(X) != P_t(X)$
为什么covariant shift会影响表现呢?这里面的分析暂且搁置.

Internal covariant shift指的就是, 在网络各层中, 上一层的输出经过仿射变换以后, 再作为下一层的输入进行计算, 这个过程中间,
就会造成上一层的输入和下一层的输入之间的covariant shift, 因为发生在网络内部, 所以称为'internal'

\subsubsection{(BN是怎么做的)}

首先BN做了两点假设: 假设输入向量/特征$\textbf{x} = [x_1, x_2, \dots, x_n]$的各个分量独立; 使用mini-batch来替代对整个数据集做norm.
第一个假设可以表述为:

\begin{align}
    \hat{x}^{(k)} = \frac{x^{(k)} - \mathbb{E}[x^{(k)}]}{\sqrt{\mathbb{D}[x^{(k)}]}}
\end{align}

注意到当我们对\textbf{x}做零均值规范化(归一化)以后, \textbf{x}服从于期望为0, 方差为1的分布, 对于像sigmoid, tanh这样的激活函数来讲,
激活值就落到了线性区域, 也就是非饱和区(non-saturated regime). 但是这样的规范化可能会损害该层的表示能力, BN中为了解决这一点,
又设置了两个可学习的参数\textbf{$\gamma, \beta$}. (3)式可以进一步写成:

\begin{align}
    y^{(k)} = \gamma^{(k)} \hat{x}^{(k)} + \beta^{(k)}
\end{align}


第二个假设, 我们对变量x的期望做了改变(也就是抽样), 令抽取的样本集上的期望和方差取代总样本的期望和方差. 但是为了处理数值稳定性, 额外添加了一个参数$\epsilon$:
注意$\hat{x}_i$中的下标指的是mini-batch中的第i个样本, $\mu$和$\sigma$是mini-batch的均值和标准差.
\begin{align}
    \hat{x}_i = \frac{x_i - \mu}{\sqrt{\sigma^2 + \epsilon}}
\end{align}

\subsubsection{训练和预测(train and inference)}

假设我们的激活函数为$\textbf{u} = f\textbf{(Wx + b)}$, 其中$\textbf{x = Wx + b}$, 我们norm的对象应该为\textbf{x}.
训练过程无需赘言, 前向和反向传播一下就行了. 重要的是inference的过程.
注意到在inference过程中间, 我们的数据是不可能以batch的形式进行归一化的, 这里我们需要对inference的输入的方差和期望进行更改.
对于训练过程中的每一个mini-batch, 我们记录下这些期望$\mu$和方差$\sigma$. 然后再为inference的输入\textbf{x}生成期望和方差:
其中infer的输入的方差为训练集方差的\href{https://baike.baidu.com/item/%E6%97%A0%E5%81%8F%E4%BC%B0%E8%AE%A1/3370664?fr=aladdin}{\textbf{无偏估计}}.

\begin{align}
    \mathbb{E}[x] & = \mathbb{E}[\mu]                  \\
    \mathbb{D}[x] & = \tfrac{m}{m-1}\mathbb{E}[\sigma]
\end{align}

那\textbf{x}可以写为:

\begin{align}
    \hat{x} = \frac{x - \mathbb{E}[x]}{\sqrt{\mathbb{D}[x]}}
\end{align}

因为$\mathbb{E}[x]$和$\mathbb{D}[x]$是固定的, 那么可以用一个仿射取替代没一层的计算(稍微减少一下计算量):

\begin{align}
    y = \tfrac{\gamma}{\sqrt{\mathbb{D}[x]}} \cdot x + (\beta - \tfrac{\mathbb{E}[x]}{\sqrt{\mathbb{D}[x]}})
\end{align}

最终我们的激活函数$\textbf{u} = f\textbf{(Wx + b)}$就可以写成$\textbf{u} = f(\text{BN}(\textbf{Wx}))$, 为什么要去掉偏置项$b$呢? 因为在求期望和方差的过程中被抵消掉了.

在pytorch的实现中, $\mathbb{E}$和$\mathbb{D}$都是通过指数移动平均计算的. 并没有算整体的均值和方差.

\subsubsection{Batch Norm的缺陷}

Batch Norm的第一个缺陷是不能用于在线学习的任务, 即当batch size为1或batch中的example相似时时, 方差为0.
第二个缺陷时Batch Norm无法用于RNN这些网络, 一些实验表明在RNN的垂直方向上(多层RNN叠加)和水平方向上(按序列展开),
表现都不是特别好. 水平方向上到后期如果有特别长的sequence时, batch norm的缺陷表现为第一个缺陷.
所有后来Hinton etc. 使用了Layer norm解决了这个问题.

\subsubsection{\href{http://arxiv.org/abs/1805.11604}{How Does Batch Normalization Help Optimization?}}

有意思的来了, 前面说BN主要解决的是internal covariate shift, 那么到底解决了吗?
上面这篇paper给出的答案是没有. \textcolor{red}{这里留个坑}


\subsection{Layer norm}

layer norm是针对BN在RNN上的缺陷提出的. 公式大体上还是2.2小节中的公式就, 
但是均值和方差的计算改变了(\textcolor{red}{可能和上面的符号不统一, 完成了整篇笔记的撰写再改吧}):
\begin{align}
    \mu = \frac{1}{d} \displaystyle \sum_{i=1}^{d} \textbf{u}_i \quad
    \sigma = \sqrt{\frac{1}{d} \displaystyle \sum_{i=1}^{d} (\textbf{u}_i - \mu)^2} 
\end{align}

如上式, Layer Norm表现为对单个仿射层向量内部的\textbf{所有元素}计算期望和方差,
而不是对一个batch内单独一个维度的所有元素计算期望和方差.
直观的可以先看下图.





\end{document}