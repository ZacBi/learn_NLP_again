\documentclass{article}
\usepackage{listings}
\usepackage{ctex}
\usepackage{amsmath}
\usepackage[a4paper, left = 2cm, right = 2cm]{geometry}
\usepackage[colorlinks, urlcolor=blue]{hyperref}
% \usepackage{ctex}

\begin{document}

\begin{titlepage}
    \title{Chapter 2 Notes}
    \author{\href{https://github.com/ZacBi}{Zac Bi}}
    \date{\today}
    \maketitle
    \pagestyle{empty}
\end{titlepage}

\section{Why we need LSTM and GRU?}

\subsection{Grandient vanish and explosion}

\paragraph*{}
首先需要谈到的问题是梯度爆炸和梯度消失。首先回顾一下SRN(simple Recurrent NN)中的Loss计算
\begin{align*}
    h_t &= f(W^h h_{t-1} + W^x x(t) + b) \\
    \hat{y}_t &= g(W^s h_t) \\
    L_t &= Loss(\hat{y}_t, y_t) \\
    L &= \sum_{t=1}^{T} L_t
\end{align*}

\paragraph*{}
现在来求Loss对于隐层单元的梯度$\frac{\partial L}{\partial W}$:
\begin{align*}
    \frac{\partial L}{\partial W^h} & = \sum_{t=1}^{T} \frac{\partial L_t}{\partial W^h} \\
    \frac{\partial L_t}{\partial W^h} & = \sum_{k=1}^{t} \frac{\partial L_t}{\partial \hat{y}_t} 
    \cdot \frac{\partial \hat{y}_t}{\partial{h_t}}
    \cdot \frac{\partial h_t}{\partial{h_k}} 
    \cdot \frac{\partial h_k}{\partial{W^h}}
    \\
\end{align*}

\paragraph*{}
其中$\frac{\partial h_t}{\partial{h_k}}$又可写成下式,最后的结果是根据Jaccobian矩阵得来,可以自己推导一下,其中$z_j = W^h h_{t-1} + W^x x(t) + b$)
\begin{align*}
    \frac{\partial h_t}{\partial{h_k}} & = \prod_{j=k+1}^{t} \frac{\partial h_j}{\partial h_{j-1}} \\
    & = \prod_{j=k+1}^{t} \frac{\partial h_j}{\partial z_j} \cdot \frac{z_j}{h_{t-1}} \\
    & = \prod_{j=k+1}^{t} diag(f'(z_t)) \cdot W^h
\end{align*}

\paragraph*{}
所以最后总的Loss对$W^h$的梯度可以写成:
\begin{align*}
    \frac{\partial L}{\partial W^h} = \sum_{t=1}^{T} \sum_{k=1}^{t} \frac{\partial L_t}{\partial \hat{y}_t}
    \cdot \frac{\partial \hat{y}_t}{\partial{h_t}}
    \cdot (\prod_{j=k+1}^{t} diag(f'(z_t)) \cdot W^h)
    \cdot \frac{\partial h_k}{\partial{W^h}}
\end{align*}





\end{document}